\section{ Fundamentals:MC understanding }\label{sec:q1}    

\begin{enumerate}
    \item True. See FRM pg 5. 
    \item False, the flight path angle is the angle between the velocity vector and the horizon and the angle of attack is the angle between the thrust and velocity vector.
    \item True, if infinitely large static stability or infinitesimally small mass product of inertia. Then momentary angle of attack will be repeatedly be zero (See LTR4 pg 25)
    \item True, with $\Psi = 2 \sin(\gamma)$, a constant $d\gamma/dt$ will satisfy the gravity turn EOM (See LTR3 pg 32)
    \item False: For maximum height, we have $V_z = V_0 - gt$. Then $Z = V_0 t - gt^2/2$. At $V_z = 0$, we have $t = V_0/g$. So $Z = V_0^2/g - V_0^2/2g = V_0^2/2g$.For maximum range, we have $V_x = V_0 \cos(45)$, where $S = V_0 \cos(45) t = V_0 t\sqrt{2}/2 $. When $Z = 0$, $V_0 - tg/2 = 0$. Thus, $2V_0/g = t$. $S$ would then be $\sqrt{2}V_0^2 /g$. 
    \item False, accuracy requirements easier met for high trajectory. (LTR8 pg 29)
    \item True, Launching west decreases inertial flight range (LTR9 pg 25)
    \item True, (in theory) increasing N would increase performance. However in reality it won't. (LTR5 pg 26)
    \item True, flight with coasting has lower culmination altitude. (LTR6 pg 11)
    \item False, increasing payload reduces velocity increases in upper stages. (LTR6 pg 28)
\end{enumerate}
