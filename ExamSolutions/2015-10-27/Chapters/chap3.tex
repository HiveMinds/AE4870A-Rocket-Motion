\section{ Unconventional launch:Design traject.throtle control law.energy gain }\label{sec:q3}  

\section*{a}
See USL pg 38

\section*{b}
We can simulate the booster separation by replicating the trajectory of the rocket and considering the following parameters and uncertainties:

\begin{enumerate}
    \item wind
    \item damage to the boosters
    \item velocity of rocket 
    \item position of rocket
\end{enumerate}

The simulation can be done for combinations of these parameters to determine the impact zone.

\section*{c}
Not covered in lectures

\section*{d}
???


\section*{e}
Extra velocity would give most performance gain (See USL pg 7), which would warrant an air launch due to the mass savings.


\section*{f}
See USL pg 11
\begin{enumerate}
    \item Limitation on launcher volume, mass, and geometry 
    \item Limitation on kinematics at separation: Mach number, altitude, flight path angle, and limitation on separation location 
    \item  Limitation on propellant category: security and safety w.r.t   . aircraft type and pilots 
    \item  Sometimes dedicated structures needed on the launcher to ensure good/stable separation, and the plane has to be at safe distance 
    \item  Requirement of specific technologies to autonomously regulate energy, pressurization, thermal behavior, etc.
\end{enumerate}