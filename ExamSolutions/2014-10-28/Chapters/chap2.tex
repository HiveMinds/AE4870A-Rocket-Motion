\section{ Multi stage rockets:mass. construction mass. V-e.id }\label{sec:q2}    
\textit{In this question we consider the launch of a $M_0 = 2,000 kg$ payload with a multi-stage rocket (N =3, number of stages). Consider the case that all three stages are identical, i.e., the effective exhaust velocity per stage is ceff = 3.2 km/s and the construction-mass ratio per stage is "$\epsilon$= 0.1. The required characteristic velocity to reach orbit is Ve,id = 12 km/s. Note: always state the used equations (without derivation).}



\subsection{(a)5 points. What is the total mass of the rocket?} 

From the slide 5.24 (velocity increment needed for a multi-stage rocket with identical stages):

\begin{equation}
	\lambda_{TOT}= \left( \frac{e^\frac{(-v_e)}{Nc_{eff}}-\epsilon}{1-\epsilon}\right)^{N} = 
	\left( \frac{e^\frac{(-12)}{3\times 3.2}-0.1}{1-0.1}\right)^{N} = 8.899\times 10^{-3} 
	\tag{SLIDE 5.24}
\end{equation}

The payload ratio for each stage:

\begin{equation}
\lambda_{i} = \lambda_{TOT}^{1/N} = 0.2072
\end{equation}

From the definition of total payload ratio: 

\begin{equation}
\lambda_{TOT} = \frac{M_u}{M_0} \ \ \ M_0 = \frac{M_u}{\lambda_{TOT}} = 224.743 t
\end{equation}

\subsection{(b) 20 points. Calculate for each of the stages, the construction massMc and the propellant mass Mp.}

From the definition of payload ratio: 

\begin{equation}
\lambda_{i} =\left. \frac{M_u}{M_0} \right\|_i =  \frac{M_{0, i+1}}{M_{0, i}}
\end{equation}

Thus, the payload mass for each stage is:

\begin{align}
M_{u,1} &= M_{0,2} =  0.2072\times 224.743 t = 46.566 t \\
M_{u,2} &= M_{0,3} = 0.2072\times 46.566 t = 9,648.66 kg \\
M_{u,3} &=  0.2072\times 9,648.66 kg = 2,000 kg 
\end{align}

Taking into acount that, for every stage, it holds the following relations:

\begin{align}
M_0 = M_u + M_c + M_p \ \ \ M_c + M_p = M_o - M_u,
\end{align}

we have a way to compute the sum of the construction and fuel masses. If we also consider the construction-mass ratio definition:

\begin{equation}
\epsilon = \frac{M_c}{M_c + M_p} \ \ \ M_c = \epsilon(M_c + M_p),
\end{equation}
as we now the ratio $r$, we can compute the construction and fuel mass in every step. The solution is given in the following table: 
\begin{table}[H]
	\centering
\begin{tabular}{|c|c|c|c|c|c|}
	\hline 
	Stage & M0 & Mu & M0-Mu & Mc & Mp \\ 
	\hline 
	1 & 224,743.69 & 46,566.89 & 178,176.80 & 17,817.68 & 160,359.12 \\
	2 & 46,566.89  & 9,648.66  & 36,918.23  & 3,691.82  & 33,226.41  \\
	3 & 9,648.66   & 2,000.00  & 7,648.66   & 764.87    & 6,883.79 \\
	\hline 
\end{tabular} 
\end{table}

\subsection{(c) 5 points. If we increase the payload mass and add boosters to the first stage, what can you say in general over the contribution to Ve,id per stage?}

SLIDES 6.29 Same problem, with even the same numbers... 

If we add boosters, the deltaV of the first stage will increase, while the deltaV of the other two stages will decrease. 