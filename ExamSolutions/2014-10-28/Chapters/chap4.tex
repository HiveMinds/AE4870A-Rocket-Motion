\section{ Launch Trajectories:Wind. constant thrust EOM vert. motion }\label{sec:q4}    

\begin{figure}[H]
	\centering
	\includegraphics[width=0.80\linewidth]{pics/screenshot003}
	\label{fig:screenshot003}
\end{figure}

\subsection{Derivation of the Mass equation}

By definition of mass flow (Slides FRM.11): 

\begin{equation}
t = \frac{M_0-M}{m} = \frac{M_0-M}{T/c_{eff}}= \frac{I_{sp}}{\Psi_0}(1-\frac{M}{M_0})
\label{eq:1}
\end{equation}

In our case: 
$$ t = \frac{I_{sp}}{\Psi_0}(1-\frac{M}{M_0}) =  \frac{300}{3}(1-\frac{200}{1000}) = 80 s$$
\subsection{Derivation of the trajectory equation}

The equation of motion while in the rail is: 

\begin{equation}
	M\frac{dV_z}{dt} = T - Mg_0 = mc_{eff} - Mg_0
\end{equation}

Operating and integrating each term, 
\begin{equation}
dV_z = \frac{T}{M} - g_0 = - c_{eff} \frac{dM}{M} - g_0 dt
\end{equation}

\begin{equation}
\frac{dh}{dt} = V_z =  c_{eff} ln(\frac{M_0}{M}) - g_0 t
\label{eq:4}
\end{equation}

If we integrate again, we obtain the altitude at each time (detailed explanation at slides FRM.11 - FRM.13):
\begin{equation}
h= \frac{I_sp^2}{\Psi_0} g_0 \left( \frac{M}{M_0}ln(\frac{M}{M_0}) + 1 - \frac{M}{M_0} \right) - \frac{1}{2}g_0 t^2 
\label{eq:2}
\end{equation}

Substituting \autoref{eq:1} into \autoref{eq:2} and inserting the numerical values for the end of the guide rail, i.e., $h=50m$, and if we call the coefficient $ x = \frac{M_0}{M_e} $  we obtain an ecuation for the mass:

\begin{equation}
h \frac{\Psi_0}{I_{sp}^2g_0}=50 \frac{3}{300^2\times 9.81} = \frac{1}{x}ln(\frac{1}{x}) + 1 - \frac{1}{x} - \frac{1}{2\Psi_0}\left(1-\frac{1}{x}\right)^2
\label{eq:3}
\end{equation}

By solving this ecuation you obtain the value for $x$ (TIP: if you got a CASIO fx-991 or similar, you can plug the equation, insert an estimate value for X, greater than 1 in this case, and press SOLVE. Otherwise, any numerical method or try and guess can work). 
$$x = \frac{M_0}{M_e} = 1.022964 \ \ \ M_e = 977.55 kg$$. 

The final time is: 

\begin{equation}
	t_e = \frac{I_{sp}}{\Psi_0}\left(1-\frac{1}{x}\right) = \frac{300}{3}\left(1-\frac{1}{ 1.022964}\right) =2.2449 s 
\end{equation}

Recalling \autoref{eq:4}, you obtain the final velocity after the rail: 

\begin{equation}
	v_e = I_{sp}g_0 ln(x) -g_0 t_e = 300\times 9.81 ln(1.022964) - 9.81\times 2.2449 = 44.798 m/s
\end{equation}

\subsection{Angle of attack and flight path angle}


\begin{figure}[H]
	\centering
	\includegraphics[width=0.3\linewidth]{pics/screenshot002}
	\caption{}
	\label{fig:screenshot002}
\end{figure}

In the instant after leaving the guide rail, the glith path angle is calculated as 

\begin{equation}
	\gamma = atan(\frac{v_e}{v_w}) = atan(\frac{44.798}{10}) = 77,41 \degree
\end{equation}