\section{ Fundamentals:MC understanding }\label{sec:q1}    

True or False

\begin{enumerate}
	\item \textit{The so-called ”Rocket Equation” (Tsiolkovsky) relates fuel mass to velocity increment.}  \textbf{FALSE.} It relates velocity increment to mass \textbf{ratio.}
	\item \textit{It is impossible to reach space with a single-stage rocket} \textbf{FALSE} Impossible to get into orbit
	\item \textit{For launch velocities up to 60\% of the circular velocity, a flat-Earth approximation will
	give about the same shooting range as the spherical-Earth approach} \textbf{TRUE} From the slide 7.26.
	\item \textit{To keep a constant acceleration, all stages in a multi-stage rocket need to be equally
	powerful} \textbf{FALSE} Latter stages are less powerful, as they carry less mass. 
	\item \textit{The accuracy requirements are easier met for a high trajectory than for a low trajectory}\textbf{TRUE} Slide 8.29
	\item \textit{Specific impulse is the thrust to weight ratio of a specific rocket} \textbf{TRUE} Definition of Specific impulse $I_{sp} = \frac{T}{Mg_0}$
	\item \textit{The burn time for constant thrust load is longer than the burn time for constant mass 
	flow if we assume identical specific impulse Isp and mass-ratio} \textbf{TRUE} Slide 2.16
	\item \textit{The influence of aerodynamic drag on rocket performance can be reduced by decreasing
	the rocket size (scaling)} \textbf{TRUE (doubt)}. Aerodinamic drag is proportional to Surface
	\item\textit{ The roll rate of a rocket should have a positive slope when crossing the pitch frequency
	to avoid so-called lock-in } \textbf{FALSE (doubt)} Not menctioned in the lecture notes while talking about lock-in. Page 48 
	\item \textit{A gravity turn maneuver is applied to reduce gravity losses} \textbf{FALSE} Its applied to use the gravity to modify the trajectory.
\end{enumerate}
